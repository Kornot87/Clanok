\documentclass[10pt,twoside,slovak,a4paper]{article}

\usepackage[slovak]{babel}
\usepackage[IL2]{fontenc}
\usepackage[utf8]{inputenc}
\usepackage{graphicx}
\usepackage{url}
\usepackage{hyperref}
\usepackage{cite}

\pagestyle{headings}

\title{Path of Exile a role-playing games} 

\author{Mário Babiar\\[2pt]
	{\small Slovenská technická univerzita v Bratislave}\\
	{\small Fakulta informatiky a informačných technológií}\\
	{\small \texttt{xbabiar@stuba.sk}}
	}

\date{\small 11. októbra 2022}

\begin{document}
\maketitle

\begin{abstract}
Článok sa zameriava na Path of Exile (PoE) od vývojárskeho štúdia videohier Grinding Gear Games (GGG) a action role-playing hry (ARPG). 1) Vysvetľuje Path of Exile, jej históriu, charakteristické herné prvky a monetizáciu. 2) Porovnáva Path of Exile s inými hrami zo žánru ARPG/RPG, ako Diablo II alebo Diablo III. 3) Hodnotí v čom Path of Exile vyniká a prečo je jedna z najlepších hier vo svojom žánri, ale aj v čom zaostáva.\\\\
\end{abstract}
 
\section{Čo je Path of Exile}
Path of Exile je bezplatná online ARPG zasadená do temného sveta z názvom Wraeclast. Hra je silno inšpirovaná Diablo II a je často označovaná ako jeho nástupca. PoE je navrhnutá na základe silnej online ekonomiky, vytvárania výzbroje a hlbokého prispôsobenia postavy rôznymi spôsobmi.

\section{Grinding Gear Games}
Grinding Gear Games alebo GGG je vývojárske štúdio videohier založené v roku 2006 zo sídlom v Novom Zélande. Medzí zakladateľov patria Chris Wilson (výkonný riaditeľ), Jonathan Rogers (technický riaditeľ), Erik Olofsson (Umelecký vedúci), Brian Weissman (Výkonný producent). Chris Wilson je výkonný riaditeľ a teda aj tvárou hry. Oznamuje každú novú expanziu do hry a pokiaľ prebiaha nejaká Q\&A s vývojármi hry, tak práve Chris Wilson odpovedá na dané otázky.

\end{document}
