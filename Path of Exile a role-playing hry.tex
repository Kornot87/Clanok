\documentclass[10pt,twoside,slovak,a4paper]{article}

\usepackage[slovak]{babel}
\usepackage[IL2]{fontenc}
\usepackage[utf8]{inputenc}
\usepackage{graphicx}
\usepackage{url}
\usepackage{hyperref}
\usepackage{cite}

\pagestyle{headings}

\title{Path of Exile a role-playing games} 

\author{Mário Babiar\\[2pt]
	{\small Slovenská technická univerzita v Bratislave}\\
	{\small Fakulta informatiky a informačných technológií}\\
	{\small \texttt{xbabiar@stuba.sk}}
	}

\date{\small 11. októbra 2022}

\begin{document}
\maketitle

\begin{abstract}
Článok sa zameriava na hru Path of Exile (PoE) od vývojárskeho štúdia videohier Grinding Gear Games (GGG) a action role-playing hry (ARPG). 1) Vysvetľuje rôzne mechaniky v hre Path of Exile, jej históriu, charakteristické herné prvky a monetizáciu. 2) Porovnáva Path of Exile s inými hrami zo žánru ARPG/RPG, ako Diablo II alebo Diablo III. 3) Hodnotí v čom Path of Exile vyniká a prečo je jedna z najlepších hier vo svojom žánri, ale aj v čom zaostáva.\\\\
\end{abstract}
 
\section{Čo je Path of Exile}
Path of Exile je bezplatná online ARPG zasadená do temného sveta z názvom Wraeclast. Hra je silno inšpirovaná Diablo II a je často označovaná aj ako jeho nástupca. PoE je navrhnutá na základe silnej online ekonomiky, vytvárania výzbroje a hlbokého prispôsobenia postavy rôznymi spôsobmi. Hra je každé 3 mesiace obohacovaná novou expanziou nazývanou liga. Liga na konci roka v decembri alebo v januári je väčšia a je zameraná na rozšírenie konca hry a pridanie nových bossov.

\section{Grinding Gear Games}
Grinding Gear Games alebo GGG je vývojárske štúdio videohier založené v roku 2006 zo sídlom v Novom Zélande, ktoré vydalo iba jednu hru, a to Path of Exile. Medzí zakladateľov patria Chris Wilson (výkonný riaditeľ), Jonathan Rogers (technický riaditeľ), Erik Olofsson (Umelecký vedúci), Brian Weissman (Výkonný producent). Chris Wilson je výkonný riaditeľ a teda aj tvárou hry. Oznamuje každú novú expanziu do hry a pokiaľ prebieha nejaká Q\&A s vývojármi hry, tak práve Chris Wilson odpovedá na dané otázky.

\section{Mechaniky v PoE}
\paragraph{Postavy}
Hráč má na výber zo 7 rôznych postáv. Každá postava začína na inej pozícii v pasívnom strome dôvodností, rôznymi atribútmi (sila, obratnosť, inteligencia) a odmenami z úloh od NPC (nehráčska postava). Neskôr v hre má každá postava na výber 3 rôzne podtriedy označované ako Ascendancy podtriedy.\cite{PoE-offisite-classes}

%\paragraph{Pasívny Strom dovedností}
%\begin{figure*}[tbh]
%\centering
%\includegraphics[scale=0.5,]{Path-of-exile-skill-tree.jpg}
%\caption{Strom dovedností}\cite{PoE-skilltree}
%\label{f:Strom dovedností}
%\end{figure*}

%\paragraph{Ascendancy}

%\paragraph{Výzbroj}

%\paragraph{Skill Gems}

%\paragraph{Akty}

%\paragraph{Ligy}

%\paragraph{Koniec hry - Atlas}

%\section{Trading}

%\section{Ekonómia}

\section{Monetizácia}
Path of Exile je zadarmo (free-to-play) hra, napriek tomu sa v komunite často vyskytuje pojem ako zaplatiť za výhru (pay-to-win). V hre existujú 3 typy mikrotransakcií, kozmetické (úprava vzhľadu postavy, upráva vzhľadov efektov kúziel alebo útokov ap.), rozšírenie úložného priestoru a balík podporovateľa, ktorý obsahuje prémiovú menu a predovšetkým exkluzívne kozmetické veci. Pay-to-win pojem sa spája práve s rozšírením úložného priestoru, a preto by som sa na túto tému chcel zamerať trochu viac.\\

Hráč ma k dispozícii inventár (5x12 políčok) a externý úložný priestor označovaný ako truhla. V truhle sa nachádzajú záložky a každá má 12x12 políčok na uloženie vecí. Každý hráč má na začiatku hry 3 takéto záložky. Jediný spôsob ako získať viac záložiek je ich kúpou za reálnu menu. Existujú aj špeciálne záložky s rôznou cenou od 4\$ do 15\$\cite{PoE-shop-stash-tabs}, ktoré majú viac priestoru a špeciálne funkcie na uľahčenie manažmentu inventára.\\

Napríklad záložka na mapy ktorá stojí 15\$ umožňuje uložiť každý typ mapy rôznych úrovní 72-krát. Na úrovni 16 je 96 rôznych máp, teda je možné uložiť až 6912 máp iba pre túto jednu úroveň. Namiesto normálnej záložky, ktorá je neprehľadná a má maximum 144 políčok, resp. máp na uloženie je možné, aby si hráč kúpil špeciálny typ záložky pre mapy, ktorá je veľmi prehľadná, má niekoľko násobne väčší priestor a má viacero špeciálnych funkcií, ako napríklad zvýraznenie mapy, ktorá ešte nebola dokončená. Takýchto špeciálnych záložiek existuje niekoľko a pokrývajú všetky typy vecí, ako meny, mapy, úlomky, lekvárové fľašky atď.\\

Dôvodom prečo sa v komunite objavuje pojem pay-to-win je, že niektoré špeciálne záložky je priam potrebné kúpiť, aby si hráč mohol naplno vychutnať hru v jej konečnej fáze, kde pre mnohých hráčov práve hra začína.\\

V štúdii Volitional Vanity(Singh Martinez, Mauricio \& Tang, Sini)\cite{Volitional-Vanity} uviedlo ako prvú kúpu záložku 6 z 9 ľudí prostredníctvom rozhovoru a 10 z 23 ľudí prostredníctvom prieskumu. Ako prvú kúpu balíka podporovateľa uviedlo 2 z 9 ľudí prostredníctvom rozhovoru a 10 z 23 ľudí prostredníctvom prieskumu. Balík podporovateľa obsahuje prémiovú menu, za ktorú za záložky dajú kúpiť a teda sa treba pozrieť na následné kúpy a čo bola ich motivácia. V tejto istej štúdii motiváciu ďalšej investície do hry uviedlo 19 z 23 ľudí práve záložky a 16 z 23 uviedlo podporu vývojárom.\\

Z tohto vidíme, že hráči kupujú hlavne záložky alebo balík podporovateľa a neskôr záložky za prémiovú menu získanú z balíka. Teda pre nových hráčov sú hlavnou motiváciou investovania do hry práve záložky, ktoré oproti normálnym záložkám ponúkajú oveľa väčší úložný priestor a veľa funkcií na uľahčenie manažovania inventára tzv. quality-of-life features.\\

Pay-to-win pojem sa v komunite vyskytuje práve preto, že tieto záložky ponúkajú obrovské výhody nielen v ľahšom manažovaní inventára, čo hráčovi umožní si hru viac vychutnať užije hru viac, ale hlavne v ušetrení času a pri predávaní veľkých kvantít typov vecí ako napr. fragmenty. Špeciálne záložky preto majú vplyv aj na to, ako rýchlo vie hráč zarobiť a teda sú označované ako pay-to-win mechanika.

%\section{Gambling v PoE}

%\section{V čom PoE vyniká}

%\section{V čom PoE zaostáva}

%\section{Záver}

\nocite{*}
\bibliographystyle{plain}
\bibliography{literatura}
\end{document}
